\documentclass{resume}

\name{Harold}{Hotelling}
\tagline{Mathematical statistician and economic theorist}
\titlextra{%
  Born: & September 29, 1895 \\
  Citizenship: & United States \\
  Wikipedia: & \href{https://en.wikipedia.org/wiki/Harold_Hotelling}{%
    \texttt{Harold\_Hotelling}}}

\begin{document}

\maketitle

\section{Education}

\begin{block}{}
  \emph{PhD Mathematics} \hfill 1924 \\
  Princeton University \hfill United States
\end{block}

\begin{block}{}
  \emph{MA Mathematics} \hfill 1921 \\
  Washington University \hfill United States
\end{block}

\begin{block}{}
  \emph{BA Journalism} \hfill 1919 \\
  Washington University \hfill United States
\end{block}

\section{Employment}

\begin{block}{%
    \emph{Professor of Mathematical Statistics} \hfill 1946--1973 \\
    University of North Carolina \hfill United States}
  \begin{itemize}
  \item Leadership of departments;
  \item North Carolina Award 1972.
  \end{itemize}
\end{block}

\begin{block}{}
  \emph{Faculty member} \hfill 1931--1946 \\
  Columbia University \hfill United States
\end{block}

\begin{block}{}
  \emph{Associate Professor of Mathematics} \hfill 1927--1931 \\
  Stanford University \hfill United States
\end{block}

\section{Achievements}

\subsection{Statistics}

\begin{description}
\item[Hotelling's T-squared distribution] A generalization of the Student's
  t-distribution in multivariate setting.
\item[Canonical correlation analysis] A way of inferring information from
  cross-covariance matrices.
\end{description}

\subsection{Economics}

\begin{itemize}
\item Eponym of Hotelling's law, Hotelling's lemma, and Hotelling's rule.
\item Influential contributions in mathematical economics.
\end{itemize}

\end{document}

%%% Local Variables:
%%% mode: latex
%%% TeX-master: t
%%% End:
